\documentclass[authoryearcitations]{UoYCSproject}
\usepackage[dvipdfm]{graphicx}
\usepackage[dvipdfm]{color}

\usepackage{amsmath, amsthm, amssymb}
\usepackage{a4wide}
\author{Jose Manuel Calderon}
\title{Quantum Dots as Biosensors}
\date{2011-September-14}
\supervisor{Prof. Samuel L. Braunstein}
\MNC
\wordcount{1337}
\abstract{ \LaTeXe\ is a document markup and processing system built
  upon Donald Knuth's type-setting system, \TeX.}
   


\begin{document} 
\maketitle
\chapter{Introduction}

\chapter{Literature Review}
In this chapter we will look at and discuss some of the literature relevant to this project. 
In order to properly understand the system being modeled we will briefly touch on quantum mechanics
and some of the necessary mathematics. In the second section we will look at numerical methods with
particular focus on the finite difference method\footnote{Also known as finite difference calculus.}.
We will also look at various matrix representations that are present in the tools used for our
model. 
 
\section{Quantum Dots}
Since being discovered\footnote{Some would say invented} in the 1970's by Leo Esaki and 
Raphael Tsu quantum dots have been of interest to physicists and engineers for numerous applications. 
Quantum dots (QDs from this point on) are structures of semiconductors ranging between ??????
RANGE HERE ??????. Their size and properties make them ideal in modelling atomic physics in a
macroscopic system \cite{Li}. For this reason, they are sometimes referred to as artificial atoms 
??CITE RELEVANT HERE??.

In this section we will look at the necessary concepts for modelling QDs, starting with basic 
quantum mechanics and then discuss some of the physical properties of quantum dots.

 
\subsection{Basics of Quantum Mechanics}

\subsection{Structure of Quantum Dots}


\section{Finite Difference Method}
The finite difference method is one of several numerical methods that can be used in approximating
solutions to differential equations\cite{Hamming}. We will work through the use of the finite 
difference method and then discuss why this method was chosen over other possible alternatives. 

\subsection{Background}
The finite difference method arose due to the inherent inability of computing devices to perform
calculations using infinitesimal calculus \cite{Hamming, zhilin}. Because taking the space
between two points to its limit is impossible on a computer, estimates of the derivative must be
used. As a continuous representation of a function is not feasible, a function is represented
by sampling values of the function at discrete points. The distance between these points will
be refered to throughout this section as $h$. 

First we will define the difference operator used in FDM and then see how the difference 
operator along with the Taylor Series can be used to provide a numerical approximation of 
functions and their derivatives. 

\subsubsection{The Difference Operator}
A fundamental concept in the FDM is that of the difference operator $\Delta$. One
way this is defined on function $f$ is as 

\begin{equation}
\label{eq:forwardDiff}
\Delta f(x) \equiv  f(x + h) - f(x)
\end{equation}

The difference operator defined in \ref{eq:forwardDiff} is known as the \emph{forward
difference operator}. Another definition of the difference operator is the \emph{central difference operator} 
which we will also discuss. 
An important property of the difference operator that is utilized in our implementation is its linearity,
 and therefore
$$\Delta (af(x) + bg(x)) = a \Delta f(x) + b\Delta g(x) $$
where $a$ and $b$ are constants. 

Just as with derivatives in the infinitesimal calculus, you can preform this operation repeatedly. Using the difference
operator twice, $\Delta [\Delta f(x) ] \equiv \Delta ^2 f(x) $, would correspond to the second derivative of the function. 

\begin{align}
 \Delta ^2 f(x)&= \Delta [\Delta f(x)]  \nonumber\\
		&= \Delta [f(x + h) - f(x)] \nonumber\\
		&= \Delta f(x + h) - \Delta f(x) \nonumber\\  
		&= \Delta f(x + h) - (f(x + h) - f(x)) \nonumber \\
		&= f(x + 2h) - f(x + h) - (f(x + h) - f(x)) \nonumber \\ 
		&= f(x + 2h) - f(x + h) - f(x + h) + f(x) \nonumber \\
 \Delta ^2 f(x)	&= f(x + 2h) - 2f(x + h) + f(x) \label{eq:deltaSquared}
\end{align}



\subsection{Use with Schr\"{o}dinger Equation}

\subsection{Sources of Error}

\chapter{Implementation}

\section{Matlab}

\subsection{Linear Algebra in Matlab}


\subsection{Particle in a box}

\section{Matrix Representations}
\subsection{Sparse Matrix Representations}


\chapter{Discussion}

\chapter{Evaluation}

\chapter{Conclusion}

\bibliography{diss}

\end{document}
