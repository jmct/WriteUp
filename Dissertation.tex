\documentclass[authoryearcitations]{UoYCSproject}
\usepackage[dvipdfm]{graphicx}
\usepackage[dvipdfm]{color}

\usepackage{amsmath, amsthm, amssymb}
\author{Jose Manuel Calderon}
\title{Quantum Dots as Biosensors}
\date{2011-September-14}
\supervisor{Prof. Samuel L. Braunstein}
\MNC
\wordcount{1337}
\abstract{ \LaTeXe\ is a document markup and processing system built
  upon Donald Knuth's type-setting system, \TeX.}
   


\begin{document} 
\maketitle
\chapter{Introduction}

\chapter{Literature Review}
In this chapter we will look at and discuss some of the literature relevant to this project. 
In order to properly understand the system being modeled we will briefly touch on quantum mechanics
and some of the necessary mathematics. In the second section we will look at numerical methods with
particular focus on the finite difference method\footnote{Also known as finite difference calculus.}.
We will also look at various matrix representations that are present in the tools used for our
model. 
 
\section{Quantum Dots}
Since being discovered\footnote{Some would say invented} in the 1970's by Leo Esaki and 
Raphael Tsu quantum dots have been of interest to physicists and engineers for numerous applications. 
Quantum dots (QDs from this point on) are structures of semiconductors ranging between ??????
RANGE HERE ??????. Their size and properties make them ideal in modelling atomic physics in a
macroscopic system \cite{Li}. For this reason, they are sometimes referred to as artificial atoms 
??CITE RELEVANT HERE??.

In this section we will look at the necessary concepts for modelling QDs, starting with basic 
quantum mechanics and then working through an analytical solution to a `particle in a box' that
will aid us in validating our model later on.

 
\subsection{Schrodinger Equation}




\section{Numerical Methods}

\section{Matrix Representations}

\chapter{Implementation}
\section{Matlab}

\subsection{Linear Algebra in Matlab}


\subsection{Particle in a box}

\subsection{Sparse Matrix Representations}


\chapter{Discussion}

\chapter{Evaluation}

\chapter{Conclusion}

\bibliography{diss}

\end{document}
