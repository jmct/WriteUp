\documentclass[authoryearcitations]{UoYCSproject}
\usepackage[dvipdfm]{graphicx}
\usepackage[dvipdfm]{color}

\usepackage{amsmath, amsthm, amssymb}
\author{Jose Manuel Calderon}
\title{Quantum Dots as Biosensors}
\date{2011-September-14}
\supervisor{Prof. Samuel L. Braunstein}
\MNC
\wordcount{1337}
\abstract{ \LaTeXe\ is a document markup and processing system built
  upon Donald Knuth's type-setting system, \TeX.}
   


\begin{document} 
\maketitle
\chapter{Introduction}

\chapter{Literature Review}

\section{Quantum Dots}
Since being discovered\footnote{Some would say invented} in the 1970's by Leo Esaki and Raphael Tsu quantum
dots have been of interest to physicists and engineers for numerous applications. 
\subsection{Schrodinger Equation}

\subsection{Particle in a box}



\section{Numerical Methods}

\chapter{Implementation}
\section{Matlab}

\subsection{Linear Algebra in Matlab}



\subsection{Sparse Matrix Representations}


\chapter{Discussion}

\chapter{Evaluation}

\chapter{Conclusion}


\end{document}
